\section{Introduction}
Since the introduction of the General Data Protection Regulation (GDPR) \cite{gdpr} in 2018 privacy impact assesments (PIA) have become a mandatory part of development cycles for products including collection or processing of personal data.

As the conduction of PIAs has not been the focus of most organizations before the GDPR, not a lot of solutions are public today. However, some approaches for specific applications exist (e.g. \cite{rfid}\cite{smart_grid}). To provide a more general, systematic solution to PIAs the PRIAM methodology \cite{de:hal-01302541} was developed. To get a better understanding about privacy risks, their evaluation, and consequences we want to take a deeper look into the PRIAM framework and its application.

In this report we will briefly present the PRIAM methodology, followed by a use case based on an example of a medical breach. In this example we will focus on the \textit{privacy harms} resulting from the breach and discuss the information PRIAM provides us about the harms and their consequences.  