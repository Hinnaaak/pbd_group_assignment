\section{Privacy Harms in an Example of a Medical Breach}
In this section we will examine a real world example of a medical privacy breach. First, we will quickly introduce the case. Then, we will identify the privacy harms based on PRIAM, leading to a brief discussion of the results.  In consideration of the scope of this report we will not provide a harm tree and risk calculation for the example, as it describes a specific, already occured \textit{feared event}.

\subsection{Scenario}
UCLA Health, the health system of the University of California, Los Angeles (UCLA), suffered a data breach in 2015, compromising sensitive personal data of approximately 4.5 million patients. In October 2014 UCLA Health started an investigation of suspicious network traffic , which did not appear to have malicious potential. However, in May 2015 a cyberattack involving the compromise of sensitive patient information was confirmed by the officials and affected patients were informed by the organization. The data compromised in the UCLA Health data breach included names, addresses, dates of birth, Social Security and medical record numbers, Medicare or health plan IDs, and medical information.\cite{uclahealthbreach}

\subsection{Privacy Harms}
We will now describe the \textit{privacy harms} for the presented scenario, based on the PRIAM model\cite[Section 3.7]{de:hal-01302541}.

Considering the scope and circumstances of the UCLA Health breach, the following examples for categories of \textit{privacy harms} can be observed:\\
\textit{Physical harms}: A patient may be subject to stalking due to their address being leaked or receive wrong medical treatment as the consequence of the compromise of their medical data;\\
\textit{Mental or psychological harms}: A patient fears misuse of their data, e.g. for identity theft, or is disadvantaged in an application due to a medical condition being disclosed;
\textit{Financial harms}: A patient has to pay a higher health insurance premium for their alleged sedentary lifestyle inferred from medical data, or have to defend them self against identity theft ;\\
\textit{Harms to dignity or reputation}: Disclosure of intimate personal habits or unhealthy lifestyle may cause a patient embarrassment;

Next, attributes in the form of \textit{victim} and \textit{intensity} can be assigned to the harms described in the categories of harms. Both attributes are measured as either \textit{Low}, \textit{Medium}, or \textit{High}, depending on how many individuals are effected for the \textit{victim} attribute, and the consequences, as well as the duration and reversibility of those consequences for the \textit{intensity} attribute. The \textit{severity} of the harm is then evaluated as the multiplication of both attributes. An exemplary result for the given scenario can be seen in Table \ref{harm_table}. It will be discussed further in the next section. For the complete description of the scale used, please refer to \cite[Section 3.7.1]{de:hal-01302541}.

\begin{table}[h!]
    \caption{Examples of harms of the UCLA Health breach and their attributes}
    \label{harm_table}
    \centering

    \begin{tabular}{c|M{4cm}|M{3cm}|c|c|c}
    \hline
        Harm & Example of event & Categories & Victims & Intensity & Severity\\
    \hline
        H.1 & Stalking & Physical & Low & Medium & Limited\\
        H.2 & Identity theft & Psychological, financial & Low & High & Significant\\
        H.3 & Increased health insurance premium & Financial & Medium & High & Maximum\\
        H.4 & Disclosure of intimate personal habits & Harm to dignity & Low & High & Significant\\
        H.5 & Wrong medical treatment & Physical & Low & High & Significant\\
        H.6 & Denial of Job & Psychologial, Financial & Low & High & Significant\\
    \hline
    \end{tabular}
\end{table}

\subsection{Discussion}
First, we should take a look at the underlying \textit{feared event} and its relevance, considering a medical data breach. An \textit{feared event} in PRIAM consists of a category, defined by privacy weaknesses, and the attributes \textit{scope} and \textit{irreversibility} \cite[Section 3.6]{de:hal-01302541}. In the context of a medical breach, PRIAM can be extended to fit the specific needs, as proposed in \cite{wairimu2022modelling}. This adopted version provides us with the category of hacking/IT incidents for our \textit{feared event}. Further, the \textit{scale} is evaluated as \textit{High}, due to the large number of affected patients, and the irreversibility is \textit{High} as well, as disclosed data can not be undisclosed again.
 
In Table \ref{harn_table}, we can observe that the \textit{victim} attribtute is most often estimated as \textit{Low}, while the \textit{intensity} is almost always scaled as \textit{High}. One factor for the \textit{intensity} evaluation is the \textit{High} \textit{irreversibility} of the \textit{feared event} that applies to the \textit{privacy harms} in this case as well. This leads to the \textit{severity} of each harm being almost exclusively rated as either \textit{Significant} or \textit{Maximum}. We can deduce that \textit{privacy harms} from a medical breach, while mostly only impacting a single individual, are of huge consequences for the affected individual. However, the \textit{High} \textit{scale} of the \textit{feared event} in this incident must be considered during assesment. 

Finally, we can take a look at the relative impact of each category of \textit{privacy harm} for our \textit{feared event}, by assigning numerical values to the \textit{severity}, as shown in \cite{wairimu2022modelling}. The results can be seen in Table \ref{related_categories}, with the average severity score being build for categories with multiple \textit{privacy harms}. It shows that the financial aspect is impacted the most for the UCLA Health breach.  

\begin{table}[h!]
    \caption{Relative impact of categories of privacy harms in the UCLA Health breach}
    \label{related_categories}
    \centering

    \begin{tabular}{c|c|c|c}
    \hline
        Category & Harms & Severity & Avg Severity Score \\
    \hline
        Physical & H.1, H.5 & Limited, Significant & 2.5 \\
        Psychological & H.2, H.6 & Significant, Significant & 3 \\
        Harm to dignity & H.4 & Significant & 3 \\
        Financial & H.2, H.3, H.6 & Significant, Maximum, Significant & 4 \\
    \hline
    \end{tabular}
\end{table}
