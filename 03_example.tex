\section{Case Study: UCLA Health Cyberattack}
In this section we will examine a real world example of a medical privacy breach. First, we will quickly introduce the case. Then, we will identify the privacy harms based on PRIAM. As this example provides a specific \textit{feared event} we will not provide harm tree for this case study to keep the scope of the report in line.\\
\subsection{Scenario}
UCLA Health, the health system of the University of California, Los Angeles (UCLA), suffered a data breach in 2015, compromising sensitive personal data of approximately 4.5 million patients. In October 2014 UCLA Health started an investigation of suspicious network traffic , which did not appear to have malicious potential. However, in May 2015 a cyberattack involving the compromise of sensitive patient information was confirmed by the officials and affected patients were informed by the organization. The data compromised in the UCLA Health data breach included names, addresses, dates of birth, Social Security and medical record numbers, Medicare or health plan IDs, and medical information.\cite{uclahealthbreach}
\subsection{Privacy Harms}
We will now describe the privacy harms for the presented scenario, based on the PRIAM model\cite[Section 3.7]{de:hal-01302541}.\\
Considering the scope and circumstances of the UCLA Health breach, the following examples for categories of privacy harms can be observed:\\
\textit{Physical harms}: A patient may be subject to stalking due to their address being leaked or receive wrong medical treatment as the consequence of the compromise of their medical data;\\
\textit{Mental or psychological harms}: A patient fears misuse of their data, e.g. for identity theft, or is disadvantaged in an application due to a medical condition being disclosed;
\textit{Financial harms}: A patient has to pay a higher health insurance premium for their alleged sedentary lifestyle inferred from medical data, or have to defend them self against identity theft ;\\
\textit{Harms to dignity or reputation}: Disclosure of intimate personal habits or unhealthy lifestyle may cause a patient embarrassment;\\
\textit{Societal harms}: The society losing faith in the healthcare system, leading to a decline in member numbers;\\
Next, attributes in the form of \textit{victim} and \textit{intensity} can be assigned to the harms described in the categories of harms. Both attributes are measured as either \textit{Low}, \textit{Medium}, or \textit{High}, depending on how many individuals are effected for the \textit{victim} attribute and the consequences, as well as the duration and reversibility of those consequences for the \textit{intensity} attribute. The \textit{severity} of the harm is then evaluated as the multiplication of both attributes. An excerpt of the result for the given example can be seen in Table \ref{harm_table}. For the complete description of the scale used, please refer to \cite[Section 3.7.1]{de:hal-01302541}.\\
\begin{table}[h!]
    \caption{Examples of harms of the UCLA Health breach and their attributes}
    \label{harm_table}
    \centering

    \begin{tabular}{c|M{4cm}|M{3cm}|c|c|c}
    \hline
        Harm & Example of event & Categories & Victims & Intensity & Severity\\
    \hline
        H.1 & Stalking & Physical & Low & Medium & Limited\\
        H.2 & Identity theft & Psychological, financial & Low & High & Significant\\
        H.3 & Increased health insurance premium & Financial & Medium & High & Maximum\\
        H.4 & Disclosure of intimate personal habits & Harm to dignity & Low & High & Significant\\
        H.5 & Loss of trust in health care system & Societal & High & Medium & Maximum\\
    \hline
    \end{tabular}
\end{table}
